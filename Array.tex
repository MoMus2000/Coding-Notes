\documentclass[24pt, a4]{article}
% To compile latex inside of vim 
% ! clear; pdlatex name_of_file.tex or "%"


% \usepackage[utf8]{inputenc}
\usepackage[parfill]{parskip}
\usepackage{listings}
\usepackage{geometry}


\title{Linked List}
\author{Mustafa Muhammad}
\date{12th November 2021}

\begin{document}

\maketitle

Checkout the other common questions like validating brackets as well ......
\section{Two sum}
Find index of two numbers that add upto the target sum
\begin{lstlisting}
class Solution:
    def twoSum(self, nums: List[int], target: int) -> List[int]:
        hash_map = {}
        for i in range(0, len(nums)):
            targ = target - nums[i]
            if targ in hash_map:
                return hash_map[targ], i
            else:
                hash_map[nums[i]] = i
                
        return -1,-1
\end{lstlisting}

\section{Best Time to Buy and Sell Stock}
Simple 1 pass through an array
Bane of my existance
\begin{lstlisting}
class Solution:
    def maxProfit(self, prices: List[int]) -> int:
        profit = 0
        buy = float('inf')
        for i in range(0, len(prices)):
            buy = min(prices[i], buy)
            profit = max(profit, prices[i]-buy)
            
        return profit
\end{lstlisting}
\section{Contains Duplicate}
\begin{lstlisting}
class Solution:
    def containsDuplicate(self, nums: List[int]) -> bool:
        dup = set()
        for i in range(0, len(nums)):
            if nums[i] in dup:
                return True
            else:
                dup.add(nums[i])
        return False
\end{lstlisting}
\section{Maximum Subarray}
\begin{lstlisting}
class Solution:
    def maxSubArray(self, nums: List[int]) -> int:
        #Kadane's algorithm repeated work
        maxSum = nums[0]
        currSum = 0
        for i in range(0, len(nums)):
            if currSum<0:
                currSum = 0
            currSum += nums[i]
            maxSum = max(maxSum, currSum)
        
        return maxSum
\end{lstlisting}
\section{Maximum Product Subarray}
\begin{lstlisting}
class Solution:
    def maxProduct(self, nums: List[int]) -> int:
        #Kadane's algorithm
        res = max(nums)
        curr_max = 1
        curr_min = 1
        
        for i in range(0, len(nums)):
            #Edge case, if 0 zero it makes eveything after it zero.
            if nums[i] == 0:
                curr_max = 1
                curr_min = 1
                continue
            
            temp = curr_max
            curr_max = max(nums[i]*curr_max, curr_min*nums[i], nums[i])
            curr_min = min(curr_min*nums[i], temp*nums[i], nums[i])
            
            res = max(res, curr_max)
        return res
\end{lstlisting}
\section{Product of Array except self}
Input: nums = [1,2,3,4]
Output: [24,12,8,6]

Do it in 0(1) without extra space
\begin{lstlisting}
class Solution:
    def productExceptSelf(self, nums: List[int]) -> List[int]:
        res = [1] * len(nums)
        prefix = 1
        for i in range(0, len(nums)):
            res[i] = prefix
            prefix *= nums[i]
        
        postfix = 1
        for j in range(len(nums)-1, -1, -1):
            res[j] *= postfix
            postfix *= nums[j]
        
        return res
\end{lstlisting}
\section{Find array in rotated sorted array}
\begin{lstlisting}
class Solution:
    def findMin(self, nums: List[int]) -> int:
        # binary search
        left = 0
        right = len(nums)-1
        
        while left <= right:
            if nums[left] <= nums[right]:
                return nums[left]
            
            mid = int(left + (right - left)/2)
            
            next = (mid+1) % len(nums)
            prev = (mid-1+len(nums)) % len(nums)
            
            if nums[mid] <= nums[prev] and nums[mid] <= nums[next]:
                return nums[mid]
        
            elif nums[left] <= nums[mid]:
                left = mid + 1
            
            elif nums[mid] <= nums[right]:
                right = mid -1
        return -1
\end{lstlisting}
\section{Search in a rotated sorted array}
\begin{lstlisting}
class Solution:
    def search(self, nums: List[int], target: int) -> int:
        def find_index(nums):
            left = 0
            right = len(nums)-1
            n = len(nums)
            
            while left <= right:
                mid = int(left + (right-left)/2)
                next = (mid + 1)%n
                prev = (mid-1+n)%n
                
                if nums[mid] <= nums[prev] and nums[mid] <= nums[next]:
                    return mid
                
                if nums[0] <= nums[mid]:
                    left = mid + 1
                
                if nums[mid] <= nums[-1]:
                    right = mid - 1
                
            return -1
        
        def binary_search(nums, target):
            left = 0 
            right = len(nums)-1
            
            while left <= right:
                mid = int(left +(right-left)/2)
                
                if nums[mid] == target:
                    return mid
                
                elif nums[mid] < target:
                    left = mid+1
                
                elif nums[mid] > target:
                    right = mid -1
            
            return -1
        
        min_index = find_index(nums)
        left_arr = binary_search(nums[0:min_index], target)
        right_arr = binary_search(nums[min_index:], target)
        
        if right_arr != -1:
            return right_arr + len(nums[0:min_index])
        
        return left_arr
\end{lstlisting}
\section{3 Sum}
Find three numbers that add upto 0
\begin{lstlisting}
class Solution:
    def threeSum(self, nums: List[int]) -> List[List[int]]:
        if len(nums) < 3:
            return None
        
        res = []
        nums = sorted(nums)
        
        for i in range(0, len(nums)):
            first = nums[i]
            if i > 0 and first == nums[i-1]:
                continue
            left = i+1
            right = len(nums)-1
            
            while left < right:
                if first + nums[left] + nums[right] < 0:
                    left += 1
                elif first+nums[left]+nums[right] > 0:
                    right -=1
                
                else:
                    res.append((first, nums[left], nums[right]))
                    left += 1
                    while nums[left] == nums[left-1] and left < right:
                        left += 1
        return res
\end{lstlisting}
\section{Container with most water}
Input: height = [1,8,6,2,5,4,8,3,7]
Output: 49

Explanation: The above vertical lines are represented by array [1,8,6,2,5,4,8,3,7]. In this case, the max area of water (blue section) the container can contain is 49.
\begin{lstlisting}
class Solution:
    def maxArea(self, height: List[int]) -> int:
        left = 0 
        right = len(height)-1
        
        max_area = 0
        
        while left < right:
            curr_height = min(height[left], height[right])
            max_area = max(max_area, (right-left)*curr_height)
            
            if height[left] < height[right]:
                left += 1
            else:
                right -=1
        
        return max_area
\end{lstlisting}

\section{Merge Sort}
Divide and conquer algorithm, to break down to a single element and recursively build up while sorting.
\begin{lstlisting}
def sort(arr):
    def merge_sort(arr):
        if len(arr) > 1:
            mid = int(len(arr)/2)

            left_arr = arr[:mid]
            right_arr = arr[mid:]

            merge_sort(left_arr)
            merge_sort(right_arr)

            i, j, k = 0, 0, 0

            while i < len(left_arr) and j < len(right_arr):
                if left_arr[i] < right_arr[j]:
                    arr[k] = left_arr[i]
                    i+=1
                else:
                    arr[k] = right_arr[j]
                    j+=1
                k+=1

            while i < len(left_arr):
                arr[k] = left_arr[i]
                i+=1
                k+=1

            while j < len(right_arr):
                arr[k] = right_arr[j]
                j+=1
                k+=1
    merge_sort(arr)
    return arr
\end{lstlisting}
\section{Quick Sort}
Select a pivot point and arrange values according to the pivot.
\begin{lstlisting}
def sort(arr):
    def quick_sort(arr):
        if len(arr) <= 1:
            return arr

        pivot = arr.pop()


        left, right = [], []

        for item in arr:
            if item < pivot:
                left.append(item)
            else:
                right.append(item)

        return quick_sort(left) + [pivot] + quick_sort(right)
    quick_sort(arr)
    return arr
\end{lstlisting}
\end{document}