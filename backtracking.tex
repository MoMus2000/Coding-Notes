\documentclass[a4]{article}

% \usepackage[utf8]{inputenc}
\usepackage[parfill]{parskip}
\usepackage{listings}
\usepackage{geometry}



\title{Backtracking}
\author{Mustafa Muhammad}
\date{25 September 2021}

\begin{document}

\maketitle

\newpage

List of Questions:
\begin{enumerate}
	\item subsets in a set
	\item permutations of a string
	\item permutation with spaces
	\item rat in a maze
	\item N queens problem (2)
	\item combination sum (3)
	\item word search 
	\item letter case permutation
\end{enumerate}

\newpage

\section{Identification}

Backtracking is an approach to find solution to a problem by searching through all possible outcomes.

It follows the same algorithm as Depth First Search.

Examples:
\begin{enumerate}
    \item Decision making (Rat in a maze, NQueens).
    \item Optimization (Graphs -- DFS).
    \item Permutations (Permutations of array or strings).
    \item Subsets.
\end{enumerate}

Identification
\begin{enumerate}
    \item Exponential Time Complexity problems (N is a small number, N $<$ 10 or N $<$ 50).
    \item Constraints.
    \item We dont have a guarantee of which outcome gives the solution.
\end{enumerate}

\newpage

\section{All Subsets in a set}
\begin{lstlisting}
class Solution:
    def subsets(self, nums: List[int]) -> List[List[int]]:
        res = []
        def perms(nums, i, subset):
            if i == 0:
                res.append(subset.copy())
                return
                
            
            perms(nums, i-1, subset)
            subset.append(nums[i-1])
            perms(nums, i-1, subset)
            subset.pop()
            return
        perms(nums, len(nums), [])
        
        return res
\end{lstlisting}
Note: append a shallow copy in the results array otherwise an empty set will show up in the result array.


\newpage

\section{Permutations of a string}
\begin{lstlisting}
class Solution:
    def find_permutation(self, S):
        string_set = set()
        def swap(arr, pos, i):
            temp = arr[pos]
            arr[pos] = arr[i]
            arr[i] = temp
            return "".join(arr)

        def helper(pos, current_string):
            if pos == len(current_string):
                string_set.add(current_string)
                return
            for i in range(pos, len(current_string)):
                current_string = swap(list(current_string), pos, i)
                helper(pos+1, current_string)
                current_string = swap(list(current_string), pos, i)

            return
        
        helper(0, S)
        
        return string_set
\end{lstlisting}
Note: backtracking step is done when are performing the recusion between two swaps.

Also Note: we use a set to avoid duplicates, we may use an array to include duplicates in our answer.

\newpage
\section{Permutations of string with spaces}
\begin{lstlisting}
class Solution:
    def permutation (self, S):        
        set_of_strings = set()

        def helper(pos, current_string):
            if pos == len(current_string):
                if current_string[-1] == " ":
                    current_string = current_string[:-1]
                set_of_strings.add(current_string)
                return

            helper(pos+1, current_string)
            #strings are immutable
            current_string = list(current_string)
            current_string.insert(pos+1, " ")
            current_string = "".join(current_string)
            helper(pos+2, current_string)
            current_string = list(current_string)
            current_string.pop(pos+1)

            return

        helper(0, S)
        return list(set_of_strings)
\end{lstlisting}
Note: Similar to generating permutation of strings, for loop is missing since we dont have to do swaps.

Also Note: Strings are immutable so convert to list before adding and removing values.

Also pay close attention to changing of index, after adding a space we increment the index by 2 and then remove the space.

\newpage
\section{Rat in a maze}
\begin{lstlisting}
class Solution:
    def findPath(self, m):
        
        paths = []
        def dfs(matrix, i, j, path):
            if i < 0 or j<0 or i > len(matrix)-1 or j > len(matrix[0])-1:
                return

            if matrix[i][j] == 0:
                return

            if i == len(matrix)-1 and j == len(matrix[0])-1:
                path.append((i,j))
                paths.append(path.copy())
                return

            matrix[i][j] = 0
            path.append((i,j))
            dfs(matrix, i+1, j, path.copy())
            dfs(matrix, i-1, j, path.copy())
            dfs(matrix, i, j+1, path.copy())
            dfs(matrix, i, j-1, path.copy())
            matrix[i][j] = 1
            path.pop()

            return

        dfs(m, 0, 0, [])

        return paths
\end{lstlisting}
Note: Simple dfs algorithm, make sure to pass copy of the array to avoid problems in popping and adding paths.

Also Note: Set visited back to one once the recursion is complete, no need for a set in this particular case.


\newpage
\section{N queens problem}
\begin{lstlisting}
class Solution:
    def nQueen(self, n):
        matrix = [[0]*n for i in range(n)]

        def is_safe(matrix, row, col):
            for i in range(col):
                if matrix[row][i] == 1:
                    return False

            for i, j in zip(range(row, -1, -1), range(col, -1, -1)):
                if matrix[i][j] == 1:
                    return False

            for i, j in zip(range(row, len(matrix), 1), range(col, -1, -1)):
                if matrix[i][j] == 1:
                    return False

            return True

        def helper(matrix, j):
            if j == len(matrix):
                return True

            for i in range(len(matrix)):

                if is_safe(matrix, i, j):
                    matrix[i][j] = 1
                    if helper(matrix.copy(), j+1):
                        return True
                    matrix[i][j] = 0

            return False

        helper(matrix, 0)
        
        return matrix
\end{lstlisting}
Hardest thing to implement is the is-safe method, otherwise the solution is trivial.

\newpage
\section{Generating all permutations of the n queens problem}
\begin{lstlisting}
class Solution:
    def totalNQueens(self, n: int) -> int:
        matrix = [[0]*n for i in range(n)]
        
        def is_safe(matrix, row, col):
            for i in range(0, col):
                if matrix[row][i] == 1:
                    return False
            for i, j in zip(range(row, -1, -1), range(col, -1, -1)):
                if matrix[i][j] == 1:
                    return False
            for i , j in zip(range(row, len(matrix)), range(col, -1, -1)):
                if matrix[i][j] == 1:
                    return False
            return True
        
        count = [0]
        def helper(matrix, j):
            if j == len(matrix):
                count[0] += 1
                return
            for i in range(0, len(matrix)):
                if is_safe(matrix, i, j):
                    matrix[i][j] = 1
                    helper(matrix.copy(), j+1)
                    matrix[i][j] = 0
            return
        
        helper(matrix, 0)
        return count[0]
\end{lstlisting}
Passing an integer as a global variable in a nested function causes problem, which is why it is passed as a list.

\newpage
\section{Combination Sum - Including duplicates}
\begin{lstlisting}
class Solution:
    def combinationSum(self, candidates: List[int], target: int)
     -> List[List[int]]:
        sol = []
        
        
        def helper(i, candidates, current_sum, target, res):
            if i > len(candidates)-1 or current_sum > target:
                return
            
            if current_sum == target:
                sol.append(res.copy())
                return
                
            
            current_sum += candidates[i]
            res.append(candidates[i])
            helper(i, candidates, current_sum, target,res)
            current_sum -= candidates[i]
            res.pop()
            helper(i+1, candidates, current_sum, target, res)
            
            return
        
        helper(0, candidates, 0, target, [])
        
        return sol
\end{lstlisting}

We are making two recursive calls - either to include or not to include, followed by the backtracking step

\newpage
\section{Combination Sum - Excluding duplicates}
\begin{lstlisting}
class Solution:
    def combinationSum2(self, candidates: List[int], target: int)
     -> List[List[int]]:
        candidates = sorted(candidates)
        sol = []
        
        
        def helper(i, current_sum, target, candidates, res):
            if current_sum == target:
                sol.append(res.copy())
                return 
            
            if i > len(candidates)-1 or current_sum > target:
                return
            
            current_sum += candidates[i]
            res.append(candidates[i])
            helper(i+1, current_sum, target, candidates, res)
            current_sum -= candidates[i]
            res.pop()
            
            #if it is the same as the last one on the current sum (see last if)

            if len(res) == 0 or res[-1] != candidates[i]:
                helper(i+1, current_sum, target, candidates, res)
            
            return
        
        helper(0, 0, target, candidates, [])
        
        return result
\end{lstlisting}

Note: Slight difference in code, instead of i and i+1 calls we only make i+1 calls and check the last if statement.

Otherwise the code does not differ that much from combination sum with duplicates.

\newpage
\section{Combination Sum - No duplicates and with specified limit on number of elements}
\begin{lstlisting}
class Solution:
    def combinationSum3(self, k: int, n: int) -> List[List[int]]:
        result = []
        thresh = [k]
        nums = [i for i in range(1,10)]
        def helper(i, nums, target, current_sum, res):
            if current_sum == target and len(res) == thresh[0]:
                result.append(res.copy())
                return
                
            if len(res) > thresh[0] or i > len(nums)-1 or current_sum > target:
                return
            
            current_sum += nums[i]
            res.append(nums[i])
            helper(i+1, nums, target, current_sum, res)
            current_sum -= nums[i]
            res.pop()
            
            if len(res) == 0 or res[-1] != nums[i]:
                helper(i+1, nums, target, current_sum, res)
            
            return
        
        helper(0, nums, n, 0, [])
        
        return result
\end{lstlisting}

Find all valid combinations of k numbers that sum up to n such that the following conditions are true:

Only numbers 1 through 9 are used.
Each number is used at most once.
Return a list of all possible valid combinations. The list must not contain the same combination twice, and the combinations may be returned in any order.

Input: k = 3, n = 7
Output: [[1,2,4]]
Explanation:
1 + 2 + 4 = 7
There are no other valid combinations.

\newpage
\section{Word Search}
\begin{lstlisting}
class Solution:
    def __init__(self):
        self.temp = ""
        self.visited = set()
    
    def exist(self, matrix: List[List[str]], word: str) -> bool:
        for i in range(0, len(matrix)):
            for j in range(0, len(matrix[0])):
                if matrix[i][j] == word:
                    return True
                
                elif word[0] == matrix[i][j] and self.dfs(matrix, i, j, 0, word):
                    return True
        
        return False
    
    
    
    def dfs(self, matrix, i, j, count, target):
        if count == len(target):
            return True
        
        if i < 0 or j < 0 or i > len(matrix)-1 or j > len(matrix[0])-1 
        or matrix[i][j] != target[count]:
            return False
        
        temp = matrix[i][j]
        matrix[i][j] = None
        
                        
        ans = self.dfs(matrix, i+1, j, count+1, target) 
        or self.dfs(matrix, i-1, j, count+1, target) 
        or self.dfs(matrix, i, j+1, count+1, target) 
        or self.dfs(matrix, i, j-1, count+1, target)
        
        matrix[i][j] = temp
                        
        return ans
\end{lstlisting}

\newpage
\section{Letter case permutation}
\begin{lstlisting}
class Solution:
    def letterCasePermutation(self, s: str) -> List[str]:
        res = []
        
        s = [letter for letter in s]
        
        def perms(pos, s):
            if pos >= len(s):
                if "".join(s) not in res:
                    res.append("".join(s))
                return
            
            s[pos] = s[pos].lower()
            perms(pos+1, s)
            s[pos] = s[pos].upper()
            perms(pos+1, s)
            
            
            return
        
        perms(0, s)
        
        return res
\end{lstlisting}

Given a string s, we can transform every letter individually to be lowercase or uppercase to create another string.

Return a list of all possible strings we could create. You can return the output in any order.

Input: s = "a1b2"
Output: ["a1b2","a1B2","A1b2","A1B2"]

\newpage
\section{Letter Combinations of a Phone Number}
\begin{lstlisting}
class Solution:
    def letterCombinations(self, digits: str) -> List[str]:
        if len(digits) == 0:
            return []
        hash_map = {
            '2': "abc",
            '3': "def",
            '4': "ghi",
            '5': "jkl",
            '6': "mno",
            '7': "pqrs",
            '8': "tuv",
            '9': "wxyz"
        }
        res = []
        def backtrack(pos, current_str):
            if len(current_str) == len(digits):
                res.append(current_str)
                return
            
            for c in hash_map[digits[pos]]:
                backtrack(pos+1, current_str+c)
            
            return
        
        backtrack(0, "")
        
        return res
\end{lstlisting}

Given a string containing digits from 2-9 inclusive, return all possible letter combinations that the number could represent. Return the answer in any order.

A mapping of digit to letters (just like on the telephone buttons) is given below. Note that 1 does not map to any letters.

Input: digits = "23"
Output: ["ad","ae","af","bd","be","bf","cd","ce","cf"]

\newpage
\section{}
\end{document}