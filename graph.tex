\documentclass[24pt, a4]{article}

% \usepackage[utf8]{inputenc}
\usepackage[parfill]{parskip}
\usepackage{listings}
\usepackage{geometry}
\usepackage{amsmath, xparse}

\title{Graph}
\author{Mustafa Muhammad}
\date{5th October 2021}

\begin{document}

\maketitle
\newpage
\begin{enumerate}
    \item{Graph basics + Representation}
    \item{Breadth First Search}
    \item{Depth First Search}
    \item{Detect cycle in undirected graph}
    \item{Topological Sort}
    \item{Dijkstra's Algorithm}
    \item{Bellman Ford ALgorithm}
    \item{Floyd Warshall's Alrogrithm}
    \item{Kosaraju's Algorithms}
    \item{Tarjan's Algorithm}
    \item{Prim's Algorithm}
    \item{Kruskal's Algorithm}
    \item{Mother Vertex in graph}
    \item{Flood Fill algorithm}
    \item{Count source to destination paths in a graph}
    \item{Rotten oranges}
    \item{Steps by knight}
    \item{Bipartite graph}
    \item{Hamiltonian path}
    \item{Travelling Salesman Problem}
    \item{Graph coloring problem}
    \item{Alien Dictionary}
    \item{Clone a graph}
    \item{Articulation point in a graph}
    \item{Bridges in a graph}
\end{enumerate}

\section{Graph Algorithms}
Graph = Set of vertices and edges

Undirected Edge = Path between two vertices with no direction

Directed Edge = Path betwen two vertices with direction

Complete graph = Every vertex has an edge to every other vertex

Indegree = How many edges go into the vertex

Outdegree = How many edges go out of a particular vertex

Bridge = An edge if removed can divide the graph into two parts

Aritculation point = Vertex if removed divides the graph into two parts

Spanning Tree = Graph with no cycles, V vertes and Edges (V-1)

\section{Representation of graphs}
Adjacency Matrix = Matrix of VxV

$\begin{bmatrix}
0 & 0 & 0 & 1 \\
0 & 0 & 0 & 1 \\
0 & 0 & 0 & 1 \\
1 & 1 & 1 & 0 
\end{bmatrix}$

In case of weighted matrix, we replace 1 with weight.

Construct = O(V$^2$)
Neighbors = O(V)
Space = O(V$^2$)

Adjacency List

0: []
1: [0, 2, 3]
2: [0, 3]
3: [0]

\section{BFS}
\begin{lstlisting}
class Solution:
    def bfsOfGraph(self, V, adj):
        res = []
        queue = []
        queue.append(0)
        visited = set()
        visited.add(V)
        while(len(queue) != 0):
            pop = queue.pop(0)
            res.append(pop)
            for node in adj[pop]:
                if node not in visited:
                    queue.append(node)
                    visited.add(node)
        return res
\end{lstlisting}
\newpage
\section{DFS}
\begin{lstlisting}
class Solution:
    def dfsOfGraph(self, V, adj):
        visited = set()
        res = []
        def dfs(node, adj):
            res.append(node)
            visited.add(node)
            for child in adj[node]:
                if child not in visited:
                    dfs(child, adj)
            return
        dfs(0, adj)
        return res
\end{lstlisting}
Time complexity of both approaches:

1, DFS TimeComplexity = O(V+E) Space = O(V+E) including stack space.

2, BFS TimeComplexity = O(V+E) Space = O(V+E)

Cycle is defined if we have a back edge in our graph
\section{Detect cycle in undirected graph}
\begin{lstlisting}
class Solution:
	def isCycle(self, V, adj):
	    visited = set()
	    def dfs_for_cycle(node, parent, adj):
	        visited.add(node)
	        for curr in adj[node]:
	            if curr not in visited:
	                if dfs_for_cycle(curr, node, adj):
	                    return True
	            elif curr != parent:
	                return True
	        return False
	    return_bool = False
	    for i in range(0, V):
	        if i not in visited:
	            return_bool = return_bool or dfs_for_cycle(i,-1,adj)
	    return return_bool
\end{lstlisting}
\newpage
\section{Detect cycle in a directed graph}
The are two parts to this question. A graph contains a cycle if it visits a 
node that it has previously visited. Therefore we keep track of two sets, one
telling us the nodes that we have currently visited and the second one telling
us the ancestors of that node.

We return true if our node is found among the ancestors, otherwise we just
traverse the subgraph using DFS.
\begin{lstlisting}
class Solution:
    def isCyclic(self, V, adj):
        
        visited = set()
        ancestor = set()
        
        def dfs(node, adj):
            visited.add(node)
            ancestor.add(node)
            
            for child in adj[node]:
                if child not in visited:
                    if dfs(child, adj):
                        return True
                elif child in ancestor:
                    return True
            
            ancestor.remove(node)
            return False
        
        res = False
        for i in range(0, V):
            if i not in visited:
                res = res or dfs(i, adj)
        return res
\end{lstlisting}
\newpage
\section{Topological Sort}
\begin{lstlisting}
Simple DFS but add node into stack before returning.

class Solution:

    def topoSort(self, V, adj):
        visited = set()
        res = []
        def top_dfs(node, adj):
            if node in visited:
                return

            visited.add(node)

            for child in adj[node]:
                top_dfs(child, adj)
            res.append(node)
            return

        for i in range(0, V):
            if i not in visited:
                top_dfs(i, adj)

        res.reverse()
        return res
\end{lstlisting}
\newpage
\section{Dijkstra's Algorithm}
It is a variation on BFS, as we are using a priority queue instead of a normal
queue and searching through frontiers. The total time complexity is 
O(E(Log(V))). 

You are given a network of n nodes, labeled from 1 to n. You are also given 
times, a list of travel times as directed edges times[i] = (ui, vi, wi), 
where ui is the source node, vi is the target node, and wi is the time it 
takes for a signal to travel from source to target.

We will send a signal from a given node k. Return the time it takes for all 
the n nodes to receive the signal. If it is impossible for all the n nodes to 
receive the signal, return -1
\begin{lstlisting}
import heapq
class Solution:
    def networkDelayTime(self, times: List[List[int]], n: int, k: int) -> int:
        #Step 1 create an adjacency list
        adjacency_map = {}
        for arr in times:
            start_node = arr[0]
            end_node = arr[1]
            weight = arr[2]
            if start_node in adjacency_map:
                adjacency_map[start_node].append((weight, end_node))
            else:
                adjacency_map[start_node] = [(weight, end_node)]        
        #initialize the min_heap
        min_heap = [(0, k)]
        t = -1
        visited = set()
        #initialize a visited set to avoid going into a cycle
        while len(min_heap) != 0:
            w1, node1 = heapq.heappop(min_heap)
            if node1 in visited:
                continue
            visited.add(node1)
            t = max(t, w1)
            if node1 in adjacency_map:
                for w2, node2 in adjacency_map[node1]:
                    if node2 not in visited:
                        heapq.heappush(min_heap, (w2+w1, node2))
        return t if len(visited) == n else -1
\end{lstlisting}
\newpage
\section{Dijkstra's returning an array of distances}
\begin{lstlisting}
import heapq
class Solution:
    def dijkstra(self, V, adj, S):
        priority_queue = [(0, S)]
        visited = set()
        distance = [-1]*V
        distance[S] = 0

        while len(priority_queue) != 0:
            w1, node1 = heapq.heappop(priority_queue)
            if node1 in visited:
                continue
            visited.add(node1)

            for node2, w2 in adj[node1]:
                new_d = w2+w1
                if distance[node2] == -1:
                    distance[node2] = new_d
                    heapq.heappush(priority_queue, (w2+w1, node2))

                elif new_d < distance[node2]:
                    distance[node2] = new_d
                    heapq.heappush(priority_queue, (w2+w1, node2))
        
        return distance
\end{lstlisting}
\newpage
\section{Bellman Ford algorithm}
Used to handle negative edge weights which can't be done using Djikstra's.

Trivial to code, involves going through all the edges and vertices.

\begin{lstlisting}
class Solution:
	def isNegativeWeightCycle(self, n, edges):
		distances = [float("inf")]*n
		distances[0] = 0
		for i in range(0, n):
		    temp = distances.copy()
		    for s, d, p in edges:
		        if distances[s] == float("inf"):
		            continue
		        if distances[s] + p < temp[d]:
		            temp[d] = distances[s]+p
		    distances = temp
        return distances[-1]
\end{lstlisting}
\section{Detecting Negative cycle (Bellman-Ford)}
To detect a negative cycle, we run the algorithm again and see if there are 
any changes from the previous distances array.
\begin{lstlisting}
class Solution:
	def isNegativeWeightCycle(self, n, edges):
		distances = [float("inf")]*n
		distances[0] = 0
		for i in range(0, n):
		    temp = distances.copy()
		    for s, d, p in edges:
		        if distances[s] == float("inf"):
		            continue
		        if distances[s] + p < temp[d]:
		            temp[d] = distances[s]+p
		    distances = temp
		for j in range(0, n):
		    for s, d, p in edges:
		        temp = distances.copy()
		        if distances[s] == float('inf'):
		            continue
		        if distances[s] + p < temp[d]:
		            return True
		return False
\end{lstlisting}
\newpage
\section{Floyd-Washall's Algorithm}
\underline {Dijkstra's}

Shortest path from one node to all nodes

\underline {Bellman-Ford}

Shortest path from one node to all nodes - negative weights allowed.

\underline {Floyd-Washall's Algorithm}

Shortest path of all vertices, negative edges are allowed.

\begin{lstlisting}
class Solution:
	def shortest_distance(self, matrix):
		for k in range(0, len(matrix)):
		    for i in range(0, len(matrix)):
		        for j in range(0, len(matrix)):
		            #-1 corresponds to an edge not existing
		            if matrix[i][k]==-1 or matrix[k][j]==-1:
		                continue
		            elif matrix[i][j] == -1:
		                matrix[i][j] = matrix[i][k] + matrix[k][j]
		            elif matrix[i][j] > matrix[i][k] + matrix[k][j]:
		                matrix[i][j] = matrix[i][k] + matrix[k][j]
		return matrix
\end{lstlisting}
\newpage
\section{Prim's algorithm}

You are given an array points representing integer coordinates of some points 
on a 2D-plane, where points[i] = [xi, yi].

Return the minimum cost to make all points connected. All points are connected
if there is exactly one simple path between any two points.

Input: points = [[0,0],[2,2],[3,10],[5,2],[7,0]]
Output: 20
Explanation:

We can connect the points as shown above to get the minimum cost of 20.
Notice that there is a unique path between every pair of points.

Every node connected together without any cycles.
\begin{lstlisting}
import heapq
class Solution:
    def minCostConnectPoints(self, points: List[List[int]]) -> int:
        adj_list = {}
        for i in range(0, len(points)):
            adj_list[i] = []
        for j in range(0, len(points)):
            x1, y1 = points[j]
            for k in range(j+1, len(points)):
                x2, y2 = points[k]
                manhattan_distance = abs(x1-x2) + abs(y1-y2)
                adj_list[k].append((manhattan_distance, j))
                adj_list[j].append((manhattan_distance, k))
        
        #Prims algorithm
        visited = set()
        res = 0
        min_heap = [(0,0)]
        while len(visited) != len(points):
            popped = heapq.heappop(min_heap)
            if popped[1] in visited:
                continue
            visited.add(popped[1])
            res+= popped[0]
            for neighbors in adj_list[popped[1]]:
                if neighbors[1] not in visited:
                    heapq.heappush(min_heap, neighbors)
        return res
\end{lstlisting}

\newpage
\section{Course Schedule}
Input: numCourses = 2, prerequisites = [[1,0],[0,1]]
Output: false

Explanation: There are a total of 2 courses to take.
To take course 1 you should have finished course 0, and to take course 0 
you should also have finished course 1. So it is impossible.

Input: numCourses = 2, prerequisites = [[1,0]]
Output: true

Explanation: There are a total of 2 courses to take. 
To take course 1 you should have finished course 0. So it is possible.
\begin{lstlisting}
class Solution:
    def canFinish(self, numCourses: int, prerequisites: List[List[int]]) -> bool:
        # step 1, create an adjacency list
        # step 2, create a visiting set
        # step 3, run dfs
        adj_list = {}
        for i in range(0, numCourses):
            adj_list[i] = []
        for req in prerequisites:
            adj_list[req[0]].append(req[1])
        visited = set()

        def dfs(course):
            if course in visited:
                return False
            if adj_list[course] == []:
                return True
            visited.add(course)
            for reqs in adj_list[course]:
                    if not dfs(reqs):
                        return False
            visited.remove(course)
            adj_list[course] = []
            return True

        for n in range(numCourses):
            if not dfs(n):
                return False
        return True
\end{lstlisting}
\newpage
\section{Course Schedule}
In short detecting cycle in graph.
\begin{lstlisting}
class Solution:
    def canFinish(self, numCourses: int, prerequisites: List[List[int]]) -> bool:
        adj_list = {}
        for i in range(0, numCourses):
            adj_list[i] = []
        for req in prerequisites:
            adj_list[req[0]].append(req[1])

        visited = set()
        anscestor = set()

        def dfs(course):
            visited.add(course)
            anscestor.add(course)
            for reqs in adj_list[course]:
                if reqs not in visited:
                    if dfs(reqs) == False:
                        return False
                elif reqs in anscestor:
                    return False
            anscestor.remove(course)
            return True

        for n in range(numCourses):
            if n not in visited:
                if dfs(n) == False:
                    return False
        return True
\end{lstlisting}
\newpage
\section{Course Schedule II}
Return a topological sort, taking care there are no cycles.
\begin{lstlisting}
class Solution:
    def findOrder(self, numCourses: int, prerequisites: List[List[int]]) -> List[int]:
        adj_map = {}
        for i in range(numCourses):
            adj_map[i] = []
        for arr in prerequisites:
            start = arr[0]
            end = arr[1]
            adj_map[start].append(end)
        
        visited = set()
        anscestor = set()
        out = []

        def dfs(node):
            visited.add(node)
            anscestor.add(node)
            for child in adj_map[node]:
                if child not in visited:
                    if dfs(child) == False:
                        return False
                elif child in anscestor:
                    return False
            out.append(node)
            anscestor.remove(node)
            return True

        for n in range(numCourses):
            if n not in visited:
                if dfs(n) == False:
                    return []
        return out
\end{lstlisting}
\newpage
\section{Number of Islands}
Input: grid = [

  ["1","1","1","1","0"],
  
  ["1","1","0","1","0"],
  
  ["1","1","0","0","0"],
  
  ["0","0","0","0","0"]
]

Output: 1

Input: grid = [

  ["1","1","0","0","0"],

  ["1","1","0","0","0"],

  ["0","0","1","0","0"],

  ["0","0","0","1","1"]
]

Output: 3
\begin{lstlisting}
class Solution:
    def numIslands(self, grid: List[List[str]]) -> int: 
        count = 0
        def dfs(grid, i, j):
            if i >= len(grid) or i<0 or j <0 or j >= len(grid[0]):
                return 
            if grid[i][j] == "0":
                return
            grid[i][j] = "0"
            dfs(grid, i+1, j)
            dfs(grid, i-1, j)
            dfs(grid, i, j+1)
            dfs(grid, i, j-1)
            return
        
        for i in range(0, len(grid)):
            for j in range(0, len(grid[0])):
                if grid[i][j] == "1":
                    dfs(grid,i,j)
                    count += 1
        return count
\end{lstlisting}
\newpage
\section{Clone Graph}
Given a reference of a node in a connected undirected graph.

Return a deep copy (clone) of the graph.

Each node in the graph contains a value (int) and a list (List[Node]) of 
its neighbors.

\begin{lstlisting}
class Node {
    
    public int val;

    public List<Node> neighbors;

}
class Solution:
    def cloneGraph(self, node: 'Node') -> 'Node':
        old_to_new = {}
        def dfs(node):
            if node == None:
                return None
            
            if node in old_to_new:
                return old_to_new[node]
            
            copy = Node(node.val)
            old_to_new[node] = copy
            for vals in node.neighbors:
                copy.neighbors.append(dfs(vals))
            return copy
        
        return dfs(node)
\end{lstlisting}
\newpage
\section{Pacific Atlantic Waterflow}
The Pacific Ocean touches the island's left and top edges, 
and the Atlantic Ocean touches the island's right and bottom edges.

The island receives a lot of rain, and the rain water can flow to neighboring 
cells directly north, south, east, and west if the neighboring cell's height 
is less than or equal to the current cell's height. Water can flow from any 
cell adjacent to an ocean into the ocean.

NOTE: The approach is to move from the pacific to areas that can be reached by
the edge blocks, perform the same for the atlantic and return the coordinates
that are intersections of both places.
\begin{lstlisting}
class Solution:
    def pacificAtlantic(self, heights: List[List[int]]) -> List[List[int]]:
        pacific = set()
        atlantic = set()
        def dfs(heights, i, j, visited, prev):
            if i<0 or j<0 or i>=len(heights) or j>= len(heights[0]):
                return
            if (i, j) in visited:
                return
            if heights[i][j] < prev:
                return
            visited.add((i,j))
            dfs(heights, i+1, j, visited, heights[i][j])
            dfs(heights, i-1, j, visited, heights[i][j])
            dfs(heights, i, j+1, visited, heights[i][j])
            dfs(heights, i, j-1, visited, heights[i][j])
            return
        #top_row
        for i in range(0, len(heights[0])):
            dfs(heights, 0, i, pacific, float('-inf'))
        #Left column
        for j in range(0, len(heights)):
            dfs(heights, j, 0, pacific, float('-inf'))
        #bottom row
        for k in range(0, len(heights[0])):
            dfs(heights, len(heights)-1, k, atlantic, float('-inf'))
        #right column
        for l in range(0, len(heights)):
            dfs(heights, l, len(heights[0])-1, atlantic, float('-inf'))
        res = []
        for i in range(0, len(heights)):
            for j in range(0, len(heights[0])):
                if (i, j) in pacific and (i,j) in atlantic:
                    res.append((i, j))
        return res
\end{lstlisting}
\newpage
\section{Network Time delay}
Dijkstra's algorithm where we find the minimum distance from source vertex "k".

You are given a network of n nodes, labeled from 1 to n. You are also given 
times, a list of travel times as directed edges times[i] = (ui, vi, wi), 
where ui is the source node, vi is the target node, and wi is the time 
it takes for a signal to travel from source to target.

We will send a signal from a given node k. Return the time it takes for 
wall the n nodes to receive the signal. If it is impossible for all the 
n nodes to receive the signal, return -1.

Input: times = [[2,1,1],[2,3,1],[3,4,1]], n = 4, k = 2
Output: 2
\begin{lstlisting}
import heapq
class Solution:
    def networkDelayTime(self, times: List[List[int]], n: int, k: int) -> int:
        adj_map = {}
        for n in range(1, n+1):
            adj_map[n] = []
        for arr in times:
            start = arr[0]
            end = arr[1]
            weight = arr[2]
            if start in adj_map:
                adj_map[start].append((weight, end))
            else:
                adj_map[start] = (weight, end)
        min_heap = [(0, k)]
        t = -1
        visited = set()
        while len(min_heap) != 0:
            city = heapq.heappop(min_heap)
            if city[1] in visited:
                continue
            visited.add(city[1])
            t = max(t, city[0])
            for neighbors in adj_map[city[1]]:
                if neighbors not in visited:
                    heapq.heappush(min_heap, (neighbors[0]+city[0], neighbors[1]))
        return t if len(visited) == n else -1
\end{lstlisting}
\newpage
\section{Word Search}
Given an m x n grid of characters board and a string word, return true if 
word exists in the grid.
The word can be constructed from letters of sequentially adjacent cells, 
where adjacent cells are horizontally or vertically neighboring. The 
same letter cell may not be used more than once.
\begin{lstlisting}
class Solution:
    def exist(self, matrix: List[List[str]], word: str) -> bool:
        visited = set()

        def dfs(matrix, i, j, count):
            if count == len(word):
                return True

            if i < 0 or j< 0 or i >= len(matrix) or j>= len(matrix[0]):
                return False

            if word[count] != matrix[i][j]:
                return False

            if (i, j) in visited:
                return False

            visited.add((i, j))

            a = dfs(matrix, i+1, j, count+1)
            b = dfs(matrix, i-1, j, count+1)
            c = dfs(matrix, i, j+1, count+1)
            d = dfs(matrix, i, j-1, count+1)
            visited.remove((i, j))

            return a or b or c or d

        for i in range(0, len(matrix)):
            for j in range(0, len(matrix[0])):
                if word[0] == matrix[i][j]:
                    if dfs(matrix, i, j, 0):
                        return True
        return False
\end{lstlisting}
\newpage
\section{Island Perimeter}
You are given row x col grid representing a map where grid[i][j] = 1 
represents land and grid[i][j] = 0 represents water.

Grid cells are connected horizontally/vertically (not diagonally). The 
grid is completely surrounded by water, and there is exactly one island 
(i.e., one or more connected land cells).

The island doesn't have "lakes", meaning the water inside isn't connected to 
the water around the island. One cell is a square with side length 1. 
The grid is rectangular, width and height don't exceed 100. 
Determine the perimeter of the island.

Input: grid = [[0,1,0,0],[1,1,1,0],[0,1,0,0],[1,1,0,0]]
Output: 16
Explanation: The perimeter is the 16 yellow stripes in the image above.
\begin{lstlisting}
class Solution:
    def islandPerimeter(self, grid: List[List[int]]) -> int:
        
        visited = set()
        
        def dfs(matrix, i, j):
            if i < 0 or j < 0 or i >= len(matrix) or j >= len(matrix[0]):
                return 1
            if matrix[i][j] == 0:
                return 1
            if (i, j) in visited:
                return 0
            visited.add((i, j))
            return dfs(matrix, i+1, j)+dfs(matrix, i-1, j)+dfs(matrix, i, j+1)+dfs(matrix, i, j-1)
        
        for i in range(0, len(grid)):
            for j in range(0, len(grid[0])):
                if grid[i][j] == 1:
                    return dfs(grid, i, j)
        return -1
\end{lstlisting}
\end{document}
